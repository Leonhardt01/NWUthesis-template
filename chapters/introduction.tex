\section{插入数学公式}
此文章是用来练习$LaTeX$的一些基本使用方法\cite{sunliguang2016jidi}。测试参考文献标注,请注意bib文件内的参考文献至少有一个被引用才可以,
先XeLaTeX编译主文件,再使用BibTeX编译一次,然后再使用XeLaTeX编译两次,一共四次\cite{sample2023}
\subsection{数学模式}
在行文中使用\$ ... \$ 可以插入行内公式。如$E=mc^2$.\\
在行文中使用$\backslash$begin\{equation \} ... $\backslash$end\{equation\} 可以插入行间公式。如:
\begin{equation}
E=mc^2.
\end{equation}
或者也可以写成equation*,这样就不会有编号。或者也可以使用$\backslash$[ 数学公式 $\backslash$],
这样也可以表示行间公式。\\在数学模式当中,上标为\^下标为\_并且只作用于之后的一个字符,如需要多个字符,请使用花括号括起来。
对于分式,如果要强制行内模式的分式显示为行间模式的大小,可以使用 $\backslash$dfrac, 
比如:$ \dfrac{1}{2} $
反之可以使用 $\backslash$tfrac
比如:$ \tfrac{1}{2} $,这个大小是与行间强制使用一样的
\subsection{运算符}
连加、连乘、极限、积分等大型运算符分别用$\backslash$sum, $\backslash$prod, $\backslash$lim, $\backslash$int
他们的上下标在行内公式中被压缩,以适应行高。我们可以用 $\backslash$limits 和 $\backslash$nolimits
%\quad means space
来强制显式地指定是否压缩这些上下标。例如:$ \sum_{i=1}^n i\quad \prod_{i=1}^n $
$ \sum\limits _{i=1}^n i\quad \prod\limits _{i=1}^n $
\[ \lim_{x\to0}x^2 \quad \int_a^b x^2 \mathrm{d}{x} \]
\[ \lim\nolimits _{x\to0}x^2\quad \int\nolimits_a^b x^2 \mathrm{d}{x} \]
\subsection{定界符}
在数学模式当中,可以使用大括号、中括号、小括号、单竖线、双竖线、单横线、双横线来表示定界符。
对于竖线$\vert$我们推荐使用$\backslash$vert来表示,而不是$\backslash \vert$,
把v改为大写V则表示双竖线
\subsection{省略号}
省略号用 $\backslash$dots, $\backslash$cdots, $\backslash$vdots, $\backslash$ddots 
等命令表示。$\backslash$dots 和 $\backslash$cdots 的纵向位置不同,前者一般用于有下标的序列。
\[ x_1,x_2,\dots ,x_n\quad 1,2,\cdots ,n\quad
\vdots\quad \ddots \]
\subsection{矩阵}
amsmath 的 pmatrix, bmatrix, Bmatrix, vmatrix, Vmatrix 
等环境可以在矩阵两边加上各种分隔符。
\begin{equation*}
\begin{pmatrix} a&b\\c&d \end{pmatrix} \quad
\begin{bmatrix} a&b\\c&d \end{bmatrix} \quad
\begin{Bmatrix} a&b\\c&d \end{Bmatrix} \quad
\begin{vmatrix} a&b\\c&d \end{vmatrix} \quad
\begin{Vmatrix} a&b\\c&d \end{Vmatrix}
\end{equation*}
当然也可以使用smallmatrix环境,可生成行内公式的小矩阵
$ ( \begin{smallmatrix} a&b\\c&d \end{smallmatrix} ) $
\subsection{长公式}
\subsubsection{不对齐}
无须对齐的长公式可以使用 multline 环境
\begin{multline*}
    x = a+b+c+{} \\
    d+e+f+g
\end{multline*}
\subsubsection{对齐}
需要对齐的公式,可以使用 aligned \textit{次环境}来实现,
它必须包含在数学环境之内,如果使用align则不必。插入\&表示我们希望对齐的位置,
而使用\{\}表示一个空白
\begin{equation*} 
    \begin{aligned}
        x ={}&a+b+c+ \\
             &d+e+f+g
    \end{aligned}
\end{equation*}
\subsection{公式组}
无需对齐的公式组可以使用 gather 环境,需要对齐的公式组可以使用 align 环境。
他们都带有编号,如果不需要编号可以使用带星花的版本。
\begin{gather}
    a = b+c+d \\
    x = y+z
\end{gather}
\begin{align*}
    a &= b+c+d \\
    x &= y+z
\end{align*}
\subsection{分段函数}
分段函数可以用cases\textit{次环境}来实现,它必须包含在数学环境之内。
\begin{equation*}
    y=
    \begin{cases}
        -\sin{x},\quad x \leqslant 0 \\
        x,\quad x>0
    \end{cases} 
\end{equation*}