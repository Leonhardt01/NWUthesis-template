% 插图
\usepackage{graphicx}

% 三线表
\usepackage{booktabs}

% 表注
\usepackage{threeparttable}

% 跨页表格
\usepackage{longtable}

% 算法
\usepackage[ruled,linesnumbered]{algorithm2e}

% 自定义字体,控制英文字符为新罗马体
\usepackage{fontspec}
\setmainfont{Times New Roman}

% 控制目录样式
\usepackage{tocloft} 

\usepackage{amsmath, amssymb, graphicx}
\numberwithin{figure}{section}
%生成pdf书签(中括号内代表给标签加编号)
\usepackage[bookmarksnumbered=true]{hyperref}

%避免图片的浮动超过Section部分
\usepackage[section]{placeins}
% 加载 enumitem 宏包以自定义 enumerate 环境
\usepackage{enumitem} 
% 用于并排显示子图
\usepackage{subcaption} 
% 指定super和square选项上标引用
\usepackage[numbers,super, square]{natbib}

\usepackage{lipsum}
% 设置页边距
\usepackage[a4paper,top=2.50cm,bottom=2.50cm,left=2.00cm,right=2.00cm,footskip=1cm]{geometry}
% 页眉页脚宏包
\usepackage{fancyhdr}
% 设置行距
\usepackage{setspace}
% 制表环境
\usepackage{array} 
% 强制图标浮动位置
\usepackage{float}

\usepackage{listings}
\lstset{breaklines=true, basicstyle=\ttfamily}

% AutoFakeBold 命令,局部打开伪粗体功能
\let\kaishu\relax                               % 清除旧定义
\newCJKfontfamily\kaishu{KaiTi}[AutoFakeBold]   % 重定义 \kaishu

% 修改图的编号格式为「节号-序号」,例如:图1-1
\renewcommand{\thefigure}{\thesection-\arabic{figure}}
\renewcommand{\thetable}{\thesection-\arabic{table}}


\fancypagestyle{plain}{ 
  \fancyhf{} % 清空默认页眉页脚
  \fancyfoot[C]{\zihao{-5} \thepage} % 页码居中,字号小五
  \renewcommand{\headrulewidth}{0pt} % 隐藏页眉线(可选)
  \renewcommand{\footrulewidth}{0pt} % 隐藏页脚线(可选)
}
% 去除页眉
\pagestyle{plain}
\fancyfoot[C]{\zihao{-5}\thepage}
\renewcommand{\headrulewidth}{0pt}  % 隐藏页眉线
\renewcommand{\footrulewidth}{0pt}  % 隐藏页脚线	
\graphicspath{{figures/}}

\captionsetup{labelformat=default,labelsep=space} %去除caption的冒号


%修改图、表汉字为黑体
\usepackage{caption}
\DeclareCaptionFont{heiti}{\heiti}
\captionsetup{labelfont={heiti}, textfont={heiti}}